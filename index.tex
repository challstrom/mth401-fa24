% Options for packages loaded elsewhere
\PassOptionsToPackage{unicode}{hyperref}
\PassOptionsToPackage{hyphens}{url}
\PassOptionsToPackage{dvipsnames,svgnames,x11names}{xcolor}
%
\documentclass[
  letterpaper,
  DIV=11,
  numbers=noendperiod]{scrreprt}

\usepackage{amsmath,amssymb}
\usepackage{iftex}
\ifPDFTeX
  \usepackage[T1]{fontenc}
  \usepackage[utf8]{inputenc}
  \usepackage{textcomp} % provide euro and other symbols
\else % if luatex or xetex
  \usepackage{unicode-math}
  \defaultfontfeatures{Scale=MatchLowercase}
  \defaultfontfeatures[\rmfamily]{Ligatures=TeX,Scale=1}
\fi
\usepackage{lmodern}
\ifPDFTeX\else  
    % xetex/luatex font selection
\fi
% Use upquote if available, for straight quotes in verbatim environments
\IfFileExists{upquote.sty}{\usepackage{upquote}}{}
\IfFileExists{microtype.sty}{% use microtype if available
  \usepackage[]{microtype}
  \UseMicrotypeSet[protrusion]{basicmath} % disable protrusion for tt fonts
}{}
\makeatletter
\@ifundefined{KOMAClassName}{% if non-KOMA class
  \IfFileExists{parskip.sty}{%
    \usepackage{parskip}
  }{% else
    \setlength{\parindent}{0pt}
    \setlength{\parskip}{6pt plus 2pt minus 1pt}}
}{% if KOMA class
  \KOMAoptions{parskip=half}}
\makeatother
\usepackage{xcolor}
\setlength{\emergencystretch}{3em} % prevent overfull lines
\setcounter{secnumdepth}{5}
% Make \paragraph and \subparagraph free-standing
\ifx\paragraph\undefined\else
  \let\oldparagraph\paragraph
  \renewcommand{\paragraph}[1]{\oldparagraph{#1}\mbox{}}
\fi
\ifx\subparagraph\undefined\else
  \let\oldsubparagraph\subparagraph
  \renewcommand{\subparagraph}[1]{\oldsubparagraph{#1}\mbox{}}
\fi


\providecommand{\tightlist}{%
  \setlength{\itemsep}{0pt}\setlength{\parskip}{0pt}}\usepackage{longtable,booktabs,array}
\usepackage{calc} % for calculating minipage widths
% Correct order of tables after \paragraph or \subparagraph
\usepackage{etoolbox}
\makeatletter
\patchcmd\longtable{\par}{\if@noskipsec\mbox{}\fi\par}{}{}
\makeatother
% Allow footnotes in longtable head/foot
\IfFileExists{footnotehyper.sty}{\usepackage{footnotehyper}}{\usepackage{footnote}}
\makesavenoteenv{longtable}
\usepackage{graphicx}
\makeatletter
\def\maxwidth{\ifdim\Gin@nat@width>\linewidth\linewidth\else\Gin@nat@width\fi}
\def\maxheight{\ifdim\Gin@nat@height>\textheight\textheight\else\Gin@nat@height\fi}
\makeatother
% Scale images if necessary, so that they will not overflow the page
% margins by default, and it is still possible to overwrite the defaults
% using explicit options in \includegraphics[width, height, ...]{}
\setkeys{Gin}{width=\maxwidth,height=\maxheight,keepaspectratio}
% Set default figure placement to htbp
\makeatletter
\def\fps@figure{htbp}
\makeatother

\KOMAoption{captions}{tableheading}
\makeatletter
\@ifpackageloaded{tcolorbox}{}{\usepackage[skins,breakable]{tcolorbox}}
\@ifpackageloaded{fontawesome5}{}{\usepackage{fontawesome5}}
\definecolor{quarto-callout-color}{HTML}{909090}
\definecolor{quarto-callout-note-color}{HTML}{0758E5}
\definecolor{quarto-callout-important-color}{HTML}{CC1914}
\definecolor{quarto-callout-warning-color}{HTML}{EB9113}
\definecolor{quarto-callout-tip-color}{HTML}{00A047}
\definecolor{quarto-callout-caution-color}{HTML}{FC5300}
\definecolor{quarto-callout-color-frame}{HTML}{acacac}
\definecolor{quarto-callout-note-color-frame}{HTML}{4582ec}
\definecolor{quarto-callout-important-color-frame}{HTML}{d9534f}
\definecolor{quarto-callout-warning-color-frame}{HTML}{f0ad4e}
\definecolor{quarto-callout-tip-color-frame}{HTML}{02b875}
\definecolor{quarto-callout-caution-color-frame}{HTML}{fd7e14}
\makeatother
\makeatletter
\@ifpackageloaded{bookmark}{}{\usepackage{bookmark}}
\makeatother
\makeatletter
\@ifpackageloaded{caption}{}{\usepackage{caption}}
\AtBeginDocument{%
\ifdefined\contentsname
  \renewcommand*\contentsname{Table of contents}
\else
  \newcommand\contentsname{Table of contents}
\fi
\ifdefined\listfigurename
  \renewcommand*\listfigurename{List of Figures}
\else
  \newcommand\listfigurename{List of Figures}
\fi
\ifdefined\listtablename
  \renewcommand*\listtablename{List of Tables}
\else
  \newcommand\listtablename{List of Tables}
\fi
\ifdefined\figurename
  \renewcommand*\figurename{Figure}
\else
  \newcommand\figurename{Figure}
\fi
\ifdefined\tablename
  \renewcommand*\tablename{Table}
\else
  \newcommand\tablename{Table}
\fi
}
\@ifpackageloaded{float}{}{\usepackage{float}}
\floatstyle{ruled}
\@ifundefined{c@chapter}{\newfloat{codelisting}{h}{lop}}{\newfloat{codelisting}{h}{lop}[chapter]}
\floatname{codelisting}{Listing}
\newcommand*\listoflistings{\listof{codelisting}{List of Listings}}
\makeatother
\makeatletter
\makeatother
\makeatletter
\@ifpackageloaded{caption}{}{\usepackage{caption}}
\@ifpackageloaded{subcaption}{}{\usepackage{subcaption}}
\makeatother
\ifLuaTeX
  \usepackage{selnolig}  % disable illegal ligatures
\fi
\usepackage{bookmark}

\IfFileExists{xurl.sty}{\usepackage{xurl}}{} % add URL line breaks if available
\urlstyle{same} % disable monospaced font for URLs
\hypersetup{
  pdftitle={MTH 401 :: Real Analysis},
  colorlinks=true,
  linkcolor={blue},
  filecolor={Maroon},
  citecolor={Blue},
  urlcolor={Blue},
  pdfcreator={LaTeX via pandoc}}

\title{MTH 401 :: Real Analysis}
\usepackage{etoolbox}
\makeatletter
\providecommand{\subtitle}[1]{% add subtitle to \maketitle
  \apptocmd{\@title}{\par {\large #1 \par}}{}{}
}
\makeatother
\subtitle{Fall 2024}
\author{}
\date{}

\begin{document}
\maketitle

\renewcommand*\contentsname{Table of contents}
{
\hypersetup{linkcolor=}
\setcounter{tocdepth}{2}
\tableofcontents
}
\bookmarksetup{startatroot}

\chapter*{Welcome!}\label{welcome}
\addcontentsline{toc}{chapter}{Welcome!}

\markboth{Welcome!}{Welcome!}

\begin{center}\rule{0.5\linewidth}{0.5pt}\end{center}

Before we get into the details of the course, I want to share with you a
few thoughts about my general approach to teaching and learning. My main
goal as the professor in this course is to help you succeed in not just
learning the material but in growing as a learner. Learning takes
effort, a willingness to try something that may not work, and the
ability to use feedback to refine your understanding. My job is to
foster an environment in which each and every one of you are supported
in these aspects.

Much of the work for this course will be done within class or small
group discussions for which we will rely on everyone in this class as a
source for feedback and support. Because of this, it's important that we
create an inclusive community that is respectful of our differences and
offers space for the boundary-setting necessary for positive
relationships to form. Our diversity is reflected by differences in
race, gender, sexuality, ability, class, religion, nationality, and
other cultural identities and material circumstances.

I am looking forward to getting to know each of you and it is my sincere
goal to make our time together (in class and out) productive and
engaging. I want you to feel comfortable coming to me with any questions
or concerns that arise, mathematical and otherwise. If you encounter any
issues that interfere with your learning, whether they are physical,
mental, emotional, economic, or otherwise, or if you experience
discrimination or mistreatment of any sort, please contact me
immediately. Other resources for support include the Chair of the Math
Department or the CAS Dean of Students.

\bookmarksetup{startatroot}

\chapter{At a Glance}\label{at-a-glance}

\subsection*{Schedule}\label{schedule}
\addcontentsline{toc}{subsection}{Schedule}

Tue-Thur, 9:45-11:10 DB 230

\subsection*{Instructor}\label{instructor}
\addcontentsline{toc}{subsection}{Instructor}

Chris Hallstrom, PhD (he / him / his). My office is Buckley Center 270.

\subsection*{Email}\label{email}
\addcontentsline{toc}{subsection}{Email}

\href{mailto:hallstro@up.edu}{\nolinkurl{hallstro@up.edu}}. Email is the
best way to contact me. I will do my best to get back to you as soon as
I can, but please be aware that I typically do not check my email in the
evenings or on weekends.

\subsection*{Webpage}\label{webpage}
\addcontentsline{toc}{subsection}{Webpage}

All course content will be posted on \href{https://learning.up.edu}{our
class Moodle page}

\subsection*{Zoom}\label{zoom}
\addcontentsline{toc}{subsection}{Zoom}

In addition to stopping by my office for questions, I am also available
(by prior arrangement) via zoom. Here is
\href{https://uportland.zoom.us/j/4098263199}{my personal zoom link}.

\subsection*{Calendly}\label{calendly}
\addcontentsline{toc}{subsection}{Calendly}

To schedule a meeting (in-person or virtual), you can check my
\href{https://calendly.com/hallstro}{Calendly scheduluer}. If you would
like to meet at time that you don't see available on Calendly, please
feel free to check with me via email!

\bookmarksetup{startatroot}

\chapter{Course Description}\label{course-description}

In this course, we will revisit many of the fundamental concepts from
Calculus I with the goal of understanding better how the basic theorems
and results follow from definitions. We will consider the question of
how we can make these concepts precise -- or what that term even means
in the context of calculus -- as well as how we can communicate our
mathematical ideas clearly. Specific topics that we will cover include a
formal definitions of the limit, continuity, and the derivative as well
as various properties and theorems involving those concepts.

We will use a method of instruction often called Inquiry Based Learning
(IBL) which is designed to engage and foster skills and habits that
working mathematicians use regularly; you will be asked to solve
problems, make conjectures, experiment, explore, create, collaborate,
and communicate your work with your peers. Rather than giving you facts
to memorize or showing you clear paths to solutions, my role is to guide
you via a sequence of carefully chosen problems through a journey of
mathematical discovery.

Throughout the semester, you will receive lists of definitions to
interpret and make sense of, as well as exercises and theorems which you
and your classmates will answer or prove. There will be very little
traditional lecture. Instead, class time will consist of student
presentations of new material. For best results, you should come to
class prepared to share your work or ideas about that day's problems.
This method of inquiry does not work nearly as well if you're looking at
a problem for the first time in-class.

You will be asked to share your solutions in class regularly. You will
also be encouraged to critique the problems (reformulating them if
needed), to generate examples and counterexamples to theorems or
conjectures, to conjecture new theorems based on what you've learned,
and to prove or disprove these conjectures. When observing another
student's presentation, it is your responsibility to follow their
argument closely and decide if they have seems reasonable. If you cannot
follow their logic, or have questions about their solution, it is your
responsibility to ask!

\begin{tcolorbox}[enhanced jigsaw, breakable, rightrule=.15mm, left=2mm, arc=.35mm, bottomrule=.15mm, colframe=quarto-callout-warning-color-frame, toprule=.15mm, leftrule=.75mm, opacityback=0, colback=white]
\begin{minipage}[t]{5.5mm}
\textcolor{quarto-callout-warning-color}{\faExclamationTriangle}
\end{minipage}%
\begin{minipage}[t]{\textwidth - 5.5mm}

A key feature of the IBL method is student \textbf{discovery} and
therefore \textbf{outside resources are not allowed.} This means that
you should not consult texts (other than the one handed out in class),
the internet, students not currently enrolled in the course, or faculty
other than myself. Consulting outside resources will only deprive
yourself of opportunities to engage with the material. You are
encouraged to work with your classmates on the problems, although for
best results you should get as far as you can on your own before
collaborating. It's important that you do not feel overwhelmed -- so
please let me know if you're stuck on a problem and I'll be happy to
give you hints.

\end{minipage}%
\end{tcolorbox}

\bookmarksetup{startatroot}

\chapter{Learning Outcomes}\label{learning-outcomes}

It will be helpful to organize the course content into

\section{Course Learning Outcomes}\label{course-learning-outcomes}

\begin{enumerate}
\def\labelenumi{\arabic{enumi}.}
\tightlist
\item
  (LO1) I can communicate mathematics orally in a clear and complete
  manner
\item
  (LO2) I can write correct and complete mathematical proofs of Real
  Analysis results using the conventions of mathematical writing
\item
  (LO3) I can independently develop correct and complete proofs of Real
  Analysis results
\item
  (LO4) I can demonstrate an understanding of the nature, approaches,
  and domain of mathematical inquiry.
\end{enumerate}

\section{Fundamental Learning
Outcomes}\label{fundamental-learning-outcomes}

\begin{itemize}
\tightlist
\item
  (FPT1) I can prove a result involving a supremum or infimum.
\item
  (FPT2) I can prove a finite limit statement using the precise
  definition.
\item
  (FPT3) I can prove an infinite limit statement using the precise
  definition.
\item
  (FPT4) I can prove a finite limit result using the precise definition.
\item
  (FPT5) I can prove an infinite limit result using the precise
  definition.
\item
  (FPT6) I can prove a result involving a limit superior or limit
  inferior.
\end{itemize}

\bookmarksetup{startatroot}

\chapter{Student Support}\label{student-support}

Throughout the semester, I expect that you will have questions that
might not get answered in class. This could happen while doing homework
or reviewing your notes, for example. Or perhaps you have a question
about the material that did not get fully resolved in class. For these
reasons, it's important that you know how to get help outside of class.

\begin{tcolorbox}[enhanced jigsaw, breakable, rightrule=.15mm, left=2mm, arc=.35mm, bottomrule=.15mm, colframe=quarto-callout-warning-color-frame, toprule=.15mm, leftrule=.75mm, opacityback=0, colback=white]
\begin{minipage}[t]{5.5mm}
\textcolor{quarto-callout-warning-color}{\faExclamationTriangle}
\end{minipage}%
\begin{minipage}[t]{\textwidth - 5.5mm}

Because of the inquiry method of the class, it's important that you
\textbf{do not} seek assistance from outside resources -- including and
especially the internet!

\end{minipage}%
\end{tcolorbox}

\section{Drop-In Student Hours}\label{drop-in-student-hours}

I have set aside the following specific times during the week that I am
available for drop-in help. You do not need to let me know you're coming
-- just stop by my office (BC 270). Many students find these drop-in
hours can be particularly helpful if you are working together with
classmates. You can work together at one of the tables down the hall
from my office and just pop in when you have questions.

\begin{itemize}
\tightlist
\item
  Mon, Wed 10:00-11:00
\item
  T-Th 2:30-4:00
\end{itemize}

\begin{tcolorbox}[enhanced jigsaw, breakable, rightrule=.15mm, left=2mm, arc=.35mm, bottomrule=.15mm, colframe=quarto-callout-note-color-frame, toprule=.15mm, leftrule=.75mm, opacityback=0, colback=white]
\begin{minipage}[t]{5.5mm}
\textcolor{quarto-callout-note-color}{\faInfo}
\end{minipage}%
\begin{minipage}[t]{\textwidth - 5.5mm}

Note: since I'm setting these times before the semester actually begins,
these times may change if my schedule changes. If it turns out that I
need to adjust them, I will let you know!

\end{minipage}%
\end{tcolorbox}

\section{Sign-Up Hours}\label{sign-up-hours}

I recognize that your schedule might not allow you to stop by during all
the posted drop-in hours. Or you might simply find it more convenient to
meet with me at a different time. If you go to my
\href{https://calendly.com/hallstro}{Calendly scheduler}, you can
sign-up for a time slot to meet, either in-person or via zoom. If there
is a specific time that works for you and you don't see it available on
the Calendly scheduler, please reach out to me via email and we will
find a time that works for your schedule.

\section{Open Door}\label{open-door}

I have the scheduled drop-in hours simply to give you some times when
you know that I'll be available -- but you are \textbf{always} welcome
to stop by my office \textbf{at any time}. Unless I'm in class or in a
meeting, I will generally be available to meet with you!

\bookmarksetup{startatroot}

\chapter{Course Structure}\label{course-structure}

\section{Daily Work}\label{daily-work}

Your standing assignment in this course is to work out and prepare
solutions / proofs for the exercises in the class notes. Keep track of
what problems we cover in class so that you can work ahead and come to
class prepared to discuss your work and share your ideas, both in small
groups and as a whole class. If you get stuck on a problem, come
prepared to ask questions.

\begin{tcolorbox}[enhanced jigsaw, breakable, rightrule=.15mm, left=2mm, arc=.35mm, bottomrule=.15mm, colframe=quarto-callout-note-color-frame, toprule=.15mm, leftrule=.75mm, opacityback=0, colback=white]
\begin{minipage}[t]{5.5mm}
\textcolor{quarto-callout-note-color}{\faInfo}
\end{minipage}%
\begin{minipage}[t]{\textwidth - 5.5mm}

I will post on Moodle a running account of what material we cover each
day so that if you happen to miss class, you can still keep track of
where we are in the notes.

\end{minipage}%
\end{tcolorbox}

\section{Written Homework}\label{written-homework}

Roughly once per week, I will ask you to hand in write-ups of selected
problems. The goal of these written assignments is to practice your
understanding of the material as well as your communication of
mathematics. I will provide feedback on your work after which you are
welcome to revise and resubmit for further feedback if you wish. You may
choose to use some of these problems as evidence of your progress in the
course.

You are welcome to write your homework assignments by hand, but you
might also choose use this opportunity to learn to use \(\LaTeX\)
(pronounced LAY-TEK). This is the typesetting system your professors use
to write documents that have math notation. There are several free ways
to use \LaTeX, the easiest of which is probably the web-based
\href{http://overleaf.com}{Overleaf}. Since it's web-based, there's
nothing to install. Look for the free student version.

\section{Synthesis Problems}\label{synthesis-problems}

Throughout the semester, we will encounter problems whose solutions
require the synthesis of multiple concepts and results from the course.
These will be more varied and more challenging and will typically
require deep thought over an extended period of time. You will likely
find that you will hit some dead ends along the way. As with all
mathematical research, the key is to learn from what doesn't work and to
persist through the difficulties. The aha moment that will eventually
come will be worth the wait! In order to encourage high standards of
work and to give you the opportunity to react to feedback, you will be
able to revise and resubmit each Synthesis Problem solution multiple
times.

\section{Due Dates}\label{due-dates}

Here is how due dates work in the non-academic world: they exist and
they're usually there for a reason, but they're usually somewhat
flexible. In this class, they are meant to help you organize your time
and to help me from getting overwhelmed with work. But if you need a
little more time (no more than a few days) to work on something, that is
usually fine. Just send me an email to let me know that your work will
be late. You do not need to provide a reason. If you need more than a
few days, you should come talk with me about it.

\section{Proof Portfolio}\label{proof-portfolio}

As we proceed through the course material, I will ask you to select
examples of your work that exemplifying your understanding and
engagement with the class content. At the end of the semester, this
portfolio will provide evidence of your learning throughout the
semester. To give me time to review your portfolio, please plan on
submitting your portfolio to me by the last day of classes, Dec.~5th.
There is no prescribed format for your portfolio -- many students create
a Google Doc that they can then share with me, but if you have other
ideas, just let me know!

\section{Class Proof Journal}\label{class-proof-journal}

Over the course of the semester, we will collaboratively create a proof
journal consisting of solutions for each problem that have been
presented in class. You can think of this as the textbook for the class,
and you may refer to these solutions (with proper attribution) in your
own work.

\section{Exams}\label{exams}

We will have a mid-semester take home exam the week of Oct.~8th which
will give you a chance to reflect on the material that we've covered up
to that point. Instead of a traditional final exam, there will have a
final reflection assignment to be completed anytime before the end of
finals week which will give you an opportunity to reflect on and
synthesize some of the main themes of the course. You will also be asked
to self-assess your learning over the course of the semester.

\bookmarksetup{startatroot}

\chapter{Writing Proofs}\label{writing-proofs}

One of the goals of this course is to hone your skill in communicating
your mathematical ideas, particularly in writing. In creating and
writing proofs, your goal is not to simply come up with a correct
argument but you must also convince someone else that you know what
you're talking about!

To best achieve this goal, you might think of your proof as a short
essay that effectively integrates mathematical work with explanation and
reasoning. Like any piece of writing, you should expect to write
multiple drafts, proofread, and edit your work. Here is an incomplete
set of guidelines you may find helpful:

\begin{itemize}
\tightlist
\item
  It is clearly written using complete sentences. In particular, all
  mathematical notation and expressions should be part of a sentence.
\item
  Writing should be neat and legible. If your handwriting is not up to
  the task, you should consider using a typesetting enviornment such as
  \(\LaTeX\)
\item
  It should probably contain more words than symbols.
\item
  It is written at a level that is appropriate for a MATH 401 audience,
  namely, members of our class who are familiar with the content of the
  course but who may not have worked on the particular problem whose
  solution you are presenting.
\item
  The proposition or claim to be proven is clearly stated as such.
\item
  Your proofs should begin with ``Proof:'' (without the quotation marks)
  and end in a way that indicates to your reader that they have reached
  the end, e.g.~with an appropriate symbol, such as \(\square\).
\item
  The proof is correct and the steps in the proof are also correct.
\item
  The proof is complete, meaning that a member of the standard Math 401
  audience (see above) can trace your reasoning from beginning to end
  and be persuaded that your proof is correct without having to
  reconstruct or guess at significant portions of your thinking.
\item
  Your writing is almost free, if not entirely free, of spelling errors.
\item
  Your writing is almost free, if not entirely free, of basic
  grammatical errors such as incomplete sentences, subject-verb
  disagreement, and misuse of punctuation.
\item
  Mathematical notation and terminology is used properly.
\end{itemize}

\bookmarksetup{startatroot}

\chapter{Grades}\label{grades}

\begin{quote}
Extrinsic motivation, which includes a desire to get better grades, is
not only different from, but often undermines, intrinsic motivation, a
desire to learn for its own sake.

-- Alfie Kohn, ``The Case Against Grades''
\end{quote}

Grades, as they are traditionally thought of, are inherently imprecise
and don't represent a full picture of your growth and learning over the
course of a semester. Worse than that, research suggests that grades
undermine the learning process in several ways:

\begin{itemize}
\item
  Grades tend to diminish interest in what you're learning.
\item
  Grades create a preference for the easiest task. In other words,
  students tend to do what they need to get a certain grade, but no
  more.
\item
  Grades tend to reduce the quality of student thinking. The moment we
  ask ``\textbf{how} am I doing?'\,' we lose track of \textbf{what}
  we're doing.
\end{itemize}

Unfortunately, I am required to submit a grade for each student at the
end of the semester but I will do what I can to de-emphasize the role of
grades in this course so that as much as possible our focus is on
learning.

\section*{Collaborative Grading}\label{collaborative-grading}
\addcontentsline{toc}{section}{Collaborative Grading}

\markright{Collaborative Grading}

Rather than giving you marks on individual assignments, I will instead
give you extensive feedback on your work. After addressing that
feedback, you're welcome to resubmit for further feedback if you wish.
Throughout the semester, I will periodically ask you to reflect
carefully on your work and to evaluate your progress. You will collect
evidence of your understanding of the course content and based on that
evidence, you will be asked to suggest a final course grade. In this
way, we will determine your your grade collaboratively.

The intention here is to help you focus on learning in a way that is
more organic, as opposed to simply working as you think you're expected
to. If this process causes more anxiety than it alleviates, please see
me at any point to confer about your progress in the course -- I'm
always happy to talk with you about your learning!

Here some of the many ways that you might demonstrate your understanding
of the course material:

\begin{itemize}
\tightlist
\item
  Submit weekly homework, including revisions that incorporate
  instructor feedback that reflects understanding of specific topics.
\item
  Successfully complete challenging synthesis problems.
\item
  Submit a proof portfolio that demonstrates your understanding through:

  \begin{itemize}
  \tightlist
  \item
    Finding and demonstrating connections between ideas.
  \item
    Constructing examples and non-examples that demonstrate
    understanding of definitions.
  \item
    Correctly using and explaining the role of axioms and definitions.
  \end{itemize}
\item
  Submit solutions to the course journal.
\item
  Provide complete solutions on quizzes or exams.
\item
  When presenting, give helpful answers to questions.
\item
  When listening to presentations, give presenters helpful feedback
  through your questions, suggestions, etc.
\item
  Read others' class journal submissions and use them to improve your
  own proof writing.
\end{itemize}

\section*{Qualitative Descriptions of
Grades}\label{qualitative-descriptions-of-grades}
\addcontentsline{toc}{section}{Qualitative Descriptions of Grades}

\markright{Qualitative Descriptions of Grades}

Here are some qualitative descriptions that I find helpful in thinking
about student grades. You might find them helpful as well.

\begin{description}
\item[A]
This grade generally indicates superior work that demonstrates a deep
understanding of the material as well as an ability to apply the
material in many unfamiliar or especially complex situations. To earn an
A, you should consistently demonstrate your deep understanding of the
material using a wide variety of methods described above.
\item[B]
This grade indicates good work that is eminently satisfactory. You
should be able to use and extend this knowledge in many situations
although you may have difficulty with particularly challenging or
unfamilar problems. To earn a B, you should consistently demonstrate
your understanding of the material using many of the methods described
above.
\item[C]
This grade indicates competent work that demonstrates an basic level of
knowledge relevant to the course. You should be able to handle most of
the more straightforward problems encountered. To earn a C, you might
consistently meet only a few of the criteria listed above or else meet
several criteria but less often.
\item[D/F]
These grades represent a fundamental breakdown of expectations. A D
represents a meaningful but unsuccessful attempt at earning a C or
above. An F represents such a severe lack of engagement, effort, or
understanding that there is no evidence of meaningful progress.
\end{description}

\section*{Engagement}\label{engagement}
\addcontentsline{toc}{section}{Engagement}

\markright{Engagement}

Although your course grade should be based on your
\textbf{understanding} of course content and not on course engagement,
in my experience these typically go hand in hand. So while engagement in
the course is not itself evidence of understanding, it does usually help
us achieve that goal.

Here are some ways that you can engage with the class:

\begin{itemize}
\tightlist
\item
  Attend class regularly
\item
  Work ahead on new problems and come to class prepared to discuss
\item
  Work to make sense of new definitions or axioms.
\item
  Ask questions - either in class or in drop-in hours.
\item
  Volunteer to present your work. If you're not comfortable sharing your
  work with the class, you can share with me during drop-in hours.
\item
  Actively participate in discussions and group work. This can mean
  sharing your work but it can also mean asking questions or helping to
  facilitate the discussion.
\item
  Support your classmates and help them succeed.
\end{itemize}

In discussing your course grade together, we may opt to add a modifier
to your grade to acccount for engagement.

\begin{itemize}
\item
  A \(\mathbf{-}\) modifier might be added to your grade if you've met
  the standard for the base grade but lack of engagment prevented you
  from doing more in the course.
\item
  A \(\mathbf{+}\) modifier might be added to your grade if you've
  engaged signficiantly in the course.
\end{itemize}

\bookmarksetup{startatroot}

\chapter{University Policies}\label{university-policies}

\section{Academic Integrity}\label{academic-integrity}

The University of Portland is a diverse academic community of learners
and scholars who are dedicated to freely sharing ideas and engaging in
respectful discussion of those ideas to discover truth. Such pursuits
require each person, whether student or faculty, to present truthfully
our own ideas and give credit to others for the ideas that they
generate. Thus, cheating on exams, copying another student's assignment,
including homework, or using the work of others without proper citation
are some examples of violating academic integrity.

Especially for written and oral assignments, students have an ethical
responsibility to properly cite the authors of any books, articles, or
other sources that they use. Students should expect to submit
assignments to Turnitin, a database that ensures assignments are
original work of the student submitting. Each discipline has guidelines
for how to give appropriate credit, and instructors will communicate the
specific guidelines for their discipline. The Clark Library also
maintains a webpage that provides citation guidelines at
\href{https://libguides.up.edu/cite}{libguides.up.edu/cite}.

The misuse of AI to shortcut course learning outcomes will be treated as
a violation of academic integrity comparable to plagiarism or cheating.
Faculty are responsible for including a written ``Course AI Policy'' in
their syllabi that clearly states what they consider appropriate and
inappropriate uses of AI in the context of their courses. Students are
responsible for using AI in ways that do not detract from the
established learning outcomes of the course. All members of the
scholarly community are responsible for demonstrating sound judgment in
discerning when and how to utilize AI in their work, upholding standards
of citation, originality, and integrity.

\section{Assessment Disclosure}\label{assessment-disclosure}

Student work products for this course may be used by the University for
educational quality assurance purposes. For reasons of confidentiality,
such examples will not include student names.

\section{Accessibility}\label{accessibility}

The University of Portland strives to make its courses and services
fully accessible to all students. Students are encouraged to discuss
with their instructors what might be most helpful in enabling them to
meet the learning goals of the course. Students who experience a
disability are encouraged to use the services of the Office for
Accessible Education Services (AES), located in the Shepard Academic
Resource Center (503-943- 8985). If you have an AES Accommodation Plan,
you should meet with your instructor to discuss how to implement your
plan in this class. Requests for alternate location for exams and/or
extended exam time should, where possible, be made two weeks in advance
of an exam, and must be made at least one week in advance of an exam.
Also, if applicable, you should meet with your instructor to discuss
emergency medical information or how best to ensure your safe evacuation
from the building in case of fire or other emergency. All information
that students provide regarding disability or accommodation is
confidential. All students are responsible for completing the required
coursework and are held to the same evaluation standards specified in
the course syllabus.

\section{Mental Health}\label{mental-health}

Anyone can experience problems with their mental health that interfere
with academic experiences and negatively impact daily life. If you or
someone you know experiences mental health challenges at UP, please
contact the \href{https://www.up.edu/counseling/}{University of Portland
Counseling Center} in the upper level of Orrico Hall (down the hill from
Franz Hall and near Mehling Hall) at 503-943-7134 or
\href{mailto:hcc@up.edu}{\nolinkurl{hcc@up.edu}}. Their services are
free and confidential. In addition, mental health consultation and
support is available through the Pilot Helpline by calling 503-943-7134
and pressing 3. The University of Portland Campus Safety Department
(503-943-4444) also has personnel trained to respond sensitively to
mental health emergencies at all hours. Remember that getting help is a
smart and courageous thing to do -- for yourself, for those you care
about, and for those who care about you. For more information on health
and wellness resources at UP go to
\href{https://www.linktr.ee/wellnessUP}{www.linktr.ee/wellnessUP}.

\section{Non-Violence}\label{non-violence}

The University of Portland is committed to fostering a safe and
respectful community free from all forms of violence. Violence of any
kind, and in particular acts of power- based personal violence, are
inconsistent with our mission. Together, all UP community members must
take a stand against violence. Learn more about what interpersonal
violence looks like, campus and community resources, UP's prevention
strategy, and what we as individuals can do to assist on the
\href{https://www.up.edu/greendot}{Green Dot website}. Further
information and reporting options may be found on the
\href{htts://www.up.edu/titleix}{Title IX website}.

\section{Ethics of Information}\label{ethics-of-information}

The University of Portland is a community dedicated to the investigation
and discovery of processes for thinking ethically and encouraging the
development of ethical reasoning in the formation of the whole person.
Using information ethically, as an element in open and honest scholarly
endeavors, involves moral reasoning to determine the right way to
access, create, distribute, and employ information, including:
considerations of intellectual property rights, fair use, information
bias, censorship, and privacy. More information can be found in the
Clark Library's guide to the
\href{https://libguides.up.edu/ethicaluse}{Ethical Use of Information}.

\section{The Learning Commons}\label{the-learning-commons}

Students may receive academic assistance through Learning Commons
tutoring services and workshops. The Co-Pilot peer tutoring program
provides students with opportunities to work with other students to get
help in writing, math, group projects, and many other courses. Schedule
an appointment to meet with a Co-Pilot (tutor) by visiting the
\href{https://www.up.edu/learningcommons}{Learning Commons website}.
Students can also meet with a Co-Pilot during drop-in hours. Check the
Learning Commons website or stop by the Learning Commons in BC 163 to
learn more about their services. Co-Pilots are a wonderful support along
your academic journey.



\end{document}
