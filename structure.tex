% Options for packages loaded elsewhere
\PassOptionsToPackage{unicode}{hyperref}
\PassOptionsToPackage{hyphens}{url}
\PassOptionsToPackage{dvipsnames,svgnames,x11names}{xcolor}
%
\documentclass[
  letterpaper,
  DIV=11,
  numbers=noendperiod]{scrreprt}

\usepackage{amsmath,amssymb}
\usepackage{iftex}
\ifPDFTeX
  \usepackage[T1]{fontenc}
  \usepackage[utf8]{inputenc}
  \usepackage{textcomp} % provide euro and other symbols
\else % if luatex or xetex
  \usepackage{unicode-math}
  \defaultfontfeatures{Scale=MatchLowercase}
  \defaultfontfeatures[\rmfamily]{Ligatures=TeX,Scale=1}
\fi
\usepackage{lmodern}
\ifPDFTeX\else  
    % xetex/luatex font selection
\fi
% Use upquote if available, for straight quotes in verbatim environments
\IfFileExists{upquote.sty}{\usepackage{upquote}}{}
\IfFileExists{microtype.sty}{% use microtype if available
  \usepackage[]{microtype}
  \UseMicrotypeSet[protrusion]{basicmath} % disable protrusion for tt fonts
}{}
\makeatletter
\@ifundefined{KOMAClassName}{% if non-KOMA class
  \IfFileExists{parskip.sty}{%
    \usepackage{parskip}
  }{% else
    \setlength{\parindent}{0pt}
    \setlength{\parskip}{6pt plus 2pt minus 1pt}}
}{% if KOMA class
  \KOMAoptions{parskip=half}}
\makeatother
\usepackage{xcolor}
\setlength{\emergencystretch}{3em} % prevent overfull lines
\setcounter{secnumdepth}{5}
% Make \paragraph and \subparagraph free-standing
\ifx\paragraph\undefined\else
  \let\oldparagraph\paragraph
  \renewcommand{\paragraph}[1]{\oldparagraph{#1}\mbox{}}
\fi
\ifx\subparagraph\undefined\else
  \let\oldsubparagraph\subparagraph
  \renewcommand{\subparagraph}[1]{\oldsubparagraph{#1}\mbox{}}
\fi


\providecommand{\tightlist}{%
  \setlength{\itemsep}{0pt}\setlength{\parskip}{0pt}}\usepackage{longtable,booktabs,array}
\usepackage{calc} % for calculating minipage widths
% Correct order of tables after \paragraph or \subparagraph
\usepackage{etoolbox}
\makeatletter
\patchcmd\longtable{\par}{\if@noskipsec\mbox{}\fi\par}{}{}
\makeatother
% Allow footnotes in longtable head/foot
\IfFileExists{footnotehyper.sty}{\usepackage{footnotehyper}}{\usepackage{footnote}}
\makesavenoteenv{longtable}
\usepackage{graphicx}
\makeatletter
\def\maxwidth{\ifdim\Gin@nat@width>\linewidth\linewidth\else\Gin@nat@width\fi}
\def\maxheight{\ifdim\Gin@nat@height>\textheight\textheight\else\Gin@nat@height\fi}
\makeatother
% Scale images if necessary, so that they will not overflow the page
% margins by default, and it is still possible to overwrite the defaults
% using explicit options in \includegraphics[width, height, ...]{}
\setkeys{Gin}{width=\maxwidth,height=\maxheight,keepaspectratio}
% Set default figure placement to htbp
\makeatletter
\def\fps@figure{htbp}
\makeatother

\KOMAoption{captions}{tableheading}
\makeatletter
\@ifpackageloaded{caption}{}{\usepackage{caption}}
\AtBeginDocument{%
\ifdefined\contentsname
  \renewcommand*\contentsname{Table of contents}
\else
  \newcommand\contentsname{Table of contents}
\fi
\ifdefined\listfigurename
  \renewcommand*\listfigurename{List of Figures}
\else
  \newcommand\listfigurename{List of Figures}
\fi
\ifdefined\listtablename
  \renewcommand*\listtablename{List of Tables}
\else
  \newcommand\listtablename{List of Tables}
\fi
\ifdefined\figurename
  \renewcommand*\figurename{Figure}
\else
  \newcommand\figurename{Figure}
\fi
\ifdefined\tablename
  \renewcommand*\tablename{Table}
\else
  \newcommand\tablename{Table}
\fi
}
\@ifpackageloaded{float}{}{\usepackage{float}}
\floatstyle{ruled}
\@ifundefined{c@chapter}{\newfloat{codelisting}{h}{lop}}{\newfloat{codelisting}{h}{lop}[chapter]}
\floatname{codelisting}{Listing}
\newcommand*\listoflistings{\listof{codelisting}{List of Listings}}
\makeatother
\makeatletter
\makeatother
\makeatletter
\@ifpackageloaded{caption}{}{\usepackage{caption}}
\@ifpackageloaded{subcaption}{}{\usepackage{subcaption}}
\makeatother
\ifLuaTeX
  \usepackage{selnolig}  % disable illegal ligatures
\fi
\usepackage{bookmark}

\IfFileExists{xurl.sty}{\usepackage{xurl}}{} % add URL line breaks if available
\urlstyle{same} % disable monospaced font for URLs
\hypersetup{
  pdftitle={Course Structure},
  colorlinks=true,
  linkcolor={blue},
  filecolor={Maroon},
  citecolor={Blue},
  urlcolor={Blue},
  pdfcreator={LaTeX via pandoc}}

\title{Course Structure}
\author{}
\date{}

\begin{document}
\maketitle

\renewcommand*\contentsname{Table of contents}
{
\hypersetup{linkcolor=}
\setcounter{tocdepth}{2}
\tableofcontents
}
\chapter{Daily Work}\label{daily-work}

Your standing assignment in this course is to work out and prepare
solutions or proofs for the exercises in the class notes. Keep track of
what problems we cover in class so that you can work ahead and come to
class prepared to discuss your work and share your ideas, both in small
groups and as a whole class. If you get stuck on a problem, come
prepared to ask questions.

I will post on Moodle a running account of what material we cover each
day so that if you do happen to miss class, you can still keep track of
where we are in the notes.

\chapter{Written Homework}\label{written-homework}

Roughly once per week, I will ask you to hand in write-ups of selected
problems. The goal of these written assignments is to demonstrate your
understanding of the material as well as your abilty to communicate that
understanding in writing. I will provide feedback on your work after
which you are welcome to revise and resubmit for further feedback if you
wish. You may choose to use some of these problems as evidence of your
progress in the course. Guidelines for written work can be found in
\textbf{?@sec-writing}.

You are welcome to write your homework assignments by hand, but you
might also choose use this opportunity to learn to use \(\LaTeX\)
(pronounced LAY-TEK). This is the typesetting system your professors use
to write documents that have math notation. There are several free ways
to use \LaTeX, the easiest of which is probably the web-based
\href{http://overleaf.com}{Overleaf}. Since it's web-based, there's
nothing to install. Look for the free student version.

\chapter{Synthesis Problems}\label{synthesis-problems}

Throughout the semester, we will encounter problems whose solutions
require the synthesis of multiple concepts and results from the course.
These will be more varied and more challenging and will typically
require deep thought over an extended period of time. You will likely
find that you will hit some dead ends along the way. As with all
mathematical research, the key is to learn from what doesn't work and to
persist through the difficulties. The aha moment that will eventually
come will be worth the wait!

\chapter{Generative AI Policy}\label{generative-ai-policy}

Given the IBL nature of the course, no AI assistance of any kind should
be used in this course. This is also true regarding the proof writing
aspect of the course -- your goal is to develop skill in communicating
mathematics through the mechanism of feedback from human readers of your
work. You should not use AI for any narrative submissions either - I'm
interested in your thoughts in your own words.

\chapter{Due Dates}\label{due-dates}

Here is how due dates work in the ``real world'' -- they exist and
they're usually there for a reason, but they're usually somewhat
flexible. In this class, they are meant to help you organize your time
and to help me from getting overwhelmed. But if you need a little more
time (no more than a few days) to work on something, that is usually
fine. Just send me an email to let me know that your work will be late.
\textbf{You do not need to provide a reason.} If you need more than a
few days, you should come talk with me about it.

\chapter{Proof Portfolio}\label{proof-portfolio}

As we proceed through the course material, I will ask you to select
examples of your work that exemplify your understanding and engagement
with the class content. At the end of the semester, this portfolio will
provide evidence of your learning throughout the semester. To give me
time to review your portfolio, please plan on submitting your portfolio
to me by the last day of classes, Dec.~5th. There is no prescribed
format for your portfolio -- many students create a Google Doc that they
can then share with me, but if you have other ideas, just let me know!

\chapter{Class Proof Journal}\label{class-proof-journal}

Over the course of the semester, we will collaboratively create a proof
journal consisting of solutions for each problem that have been
presented in class. You can think of this as a textbook for the class,
and you may refer to these solutions (with proper attribution) in your
own work.

\chapter{Exams}\label{exams}

We will have a mid-semester take home exam the week of Oct.~8th which
will give you a chance to demonstrate understanding of the material that
we've covered up to that point. Whether or not you include this in your
portfolio is up to you.

We will not have a traditional final exam. Instead, we will have a final
reflection assignment to be completed during finals week which will give
you an opportunity to reflect on and synthesize some of the main themes
of the course. You will also be asked to self-assess your learning over
the course of the semester.



\end{document}
